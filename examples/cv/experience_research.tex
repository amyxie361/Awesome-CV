%-------------------------------------------------------------------------------
%	SECTION TITLE
%-------------------------------------------------------------------------------
\cvsection{Projects}


%-------------------------------------------------------------------------------
%	CONTENT
%-------------------------------------------------------------------------------
\begin{cventries}

%---------------------------------------------------------

  \cventry
    {University of Waterloo}
    {{\href{https://github.com/amyxie361/CS651_Paper_recommendation}{\underline{Paper Recommendation using GraphX}}}} % Job title
    {Advisor: Prof. Jimmy Lin} % Location
    {Jan. 2019 - Apr. 2019} % Date(s)
    {
      \begin{cvitems} % Description(s) of tasks/responsibilities
        \item {Applied \textbf{GraphX} to build an academic paper recommendation system.}
        \item {Implemented PageRank, keyword filtering, and pattern finding algorithms in GraphX and compared the framework against \textbf{MapReduce} on \textbf{Hadoop}.}
        \item {Applied the algorithm on a citation network to give recommendations of papers, taking users’ interest into account. }
      \end{cvitems}
    }
    
    \cventry
    {University of Waterloo}
    {{\href{https://github.com/amyxie361/CS886}{\underline{Contextual Decomposition for Rationalizing LSTM Predictions}}}} % Job title
    {Advisor: {\href{https://cs.uwaterloo.ca/~y328yu/}{\underline{Prof. Yaoliang Yu}}}} % Location
    {Sept. 2018 - Nov. 2018} % Date(s)
    {
      \begin{cvitems} % Description(s) of tasks/responsibilities
        \item {Decomposed and analyzed LSTM model in token level to understand the effectiveness source of the model based on entity detection task with \textbf{PyTorch}.}
        \item {Implemented and modified multi-view concept to understand the learned weight in the two-directions of LSTM model.}
        \item {Traced the source of the effective source of LSTM models and concluded the most effectiveness comes from embedding.}
      \end{cvitems}
    }
    
      \cventry
     {Shanghai Key Laboratory of Intelligent Information Processing, Fudan University} 
    {BioASQ: Question Answering Based on Biomedical Paper Database} % Job title
    {Advisor: Prof. Shanfeng Zhu} % Location
    {July 2017 - Apr. 2018} % Date(s)
    {
      \begin{cvitems} % Description(s) of tasks/responsibilities
	\item {Introduce occurrence possibility of words as the representation of answers for question answering.}
	%\item {Adapted source code from \textbf{TensorFlow} for text processing including applying skip gram algorithm for word embedding matrix training.}
	\item {Constructed Bidirectional LSTM Recurrent Neural Networks under \textbf{Tensorflow} and introduce attention mechanism based on questions.}
	\item {Adjusted the model and achieved average Factoid MRR of 0.1615, List F measure of 0.1353.}
      \end{cvitems}
    }

\end{cventries}
