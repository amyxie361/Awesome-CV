%-------------------------------------------------------------------------------
%	SECTION TITLE
%-------------------------------------------------------------------------------
\cvsection{Research Experiences}


%-------------------------------------------------------------------------------
%	CONTENT
%-------------------------------------------------------------------------------
\begin{cventries}

%---------------------------------------------------------

 \cventry
    {University of Waterloo \& RSVP.ai}
    {Chatbot Module Development - Research Assistant} % Job title
    {Advisor:Prof. Ming Li, and Prof. Jimmy Lin} % Location
    {Sept. 2018 - Present} % Date(s)
    {
      \begin{cvitems} % Description(s) of tasks/responsibilities
        \item {Constructed an end-to-end question answering system that integrates BERT (\textbf{Tensorflow}) with the open-source Anserini (a \textbf{Lucene} IR toolkit) information retrieval toolkit both in English and Chinese and create new state of the art.}
        \item {Transferred code from \textbf{Tensorflow} implementation to \textbf{PyTorch}.}
        \item {Improved the system's performance by 10\% exact match rate on SQuAD 1.1 under open-domain setting using text augmentation and established new baselines on WebQuestion, CMRC and DRCD datasets.}
	\item {Tested the system's performance with\textbf{Elastic Search API} and provided real-time online service.}
        \item {Applied the system to domain specific document information retrieval.}
        \item {Applied a BERT based named entity recognition model to contract key information extraction. Implemented and compared bi-directional LSTM, VAE and GPT-2 as paraphrase generation models.}
      \end{cvitems}
    }
 
   \cventry
    {University of Waterloo}
    {Understanding How Human Generate Questions } % Job title
    {Advisor: Prof. Ori Friedman} % Location
    {Sept. 2019 - Presernt} % Date(s)
    {
      \begin{cvitems} % Description(s) of tasks/responsibilities
        \item {Surveyed the source of curiosity from psychology perspective.}
        \item {Surveyed the theory on linguistic perspective on question language.}
        \item {Analyzed the language features on two datasets: SQuAD and Quora Quesition Pair. }
      \end{cvitems}
    }
    
  \cventry
    {University of Waterloo}
    {{\href{https://github.com/amyxie361/CS651_Paper_recommendation}{\underline{Paper Recommendation using GraphX}}}} % Job title
    {Advisor: Prof. Jimmy Lin} % Location
    {Jan. 2019 - April 2019} % Date(s)
    {
      \begin{cvitems} % Description(s) of tasks/responsibilities
        \item {Applied \textbf{GraphX} to build an academic paper recommendation system.}
        \item {Implemented PageRank, keyword filtering, and pattern finding algorithms in GraphX and compared the framework against \textbf{MapReduce} on \textbf{Hadoop}.}
        \item {Applied the algorithm on a citation network to give recommendations of papers, taking users’ interest into account. }
      \end{cvitems}
    }

\cventry
    {University of Waterloo}
    {{\href{https://github.com/amyxie361/CS886}{\underline{Contextual Decomposition for Rationalizing LSTM Predictions}}}} % Job title
    {Advisor: P{\href{https://cs.uwaterloo.ca/~y328yu/}{\underline{Prof. Yaoliang Yu}}}} % Location
    {Sept. 2018 - Nov 2018} % Date(s)
    {
      \begin{cvitems} % Description(s) of tasks/responsibilities
        \item {Decomposed and analyzed LSTM model in token level to understand the effectiveness source of the model based on entity detection task with \textbf{PyTorch}.}
        \item {Implemented and modified multi-view concept to understand the learned weight in the two-directions of LSTM model.}
        \item {Traced the source of the effective source of LSTM models and concluded the most effectiveness comes from embedding.}
      \end{cvitems}
    }
    
    \cventry
    {Machine Learning Algorithm Intership} % Job title
    {Yitu-Tech} % Organization
    {Shanghai, China} % Location
    {Feb. 2018 - June 2018} % Date(s)
    {
      \begin{cvitems} % Description(s) of tasks/responsibilities
       \item {Worked on the car detection project.}
        \item {Improved the  \textbf{Single Shot MultiBox Detector} model for object detection on car detection task.}
        \item {Implemented the HOG-SVM for digital recognition in car license detection.}
      \end{cvitems}
    }
    
      \cventry
     {Shanghai Key Laboratory of Intelligent Information Processing, Fudan University} 
    {BioASQ: Question Answering Based on Biomedical Paper Database} % Job title
    {Advisor: Prof . Shanfeng Zhu} % Location
    {July. 2017 - April. 2018} % Date(s)
    {
      \begin{cvitems} % Description(s) of tasks/responsibilities
	\item {Introduce occurrence possibility of words as the representation of answers for question answering.}
	\item {Adapted source code from TensorFlow for text processing including applying skip gram algorithm for word embedding matrix training.}
	\item {Constructed Bidirectional LSTM Recurrent Neural Networks under \textbf{Tensorflow} framework and introduced attention mechanism based on questions.}
	\item {Adjusted the model and achieved average Factoid MRR of 0.1615, List F measure of 0.1353.}
      \end{cvitems}
    }

    \cventry
     {Fudan University} 
    {Place of Interest Prediction on Weibo Dataset - NLPCC 2017 shared task} % Job title
    {Advisor:Prof Zhongyu Wei} % Location
    {April. 2017 - June. 2017} % Date(s)
    {
      \begin{cvitems} % Description(s) of tasks/responsibilities
	\item {Introduced bipartite graph structure for similarity measuring according to user visiting records. Clustered users and spots based on semantic similarity of tweets, Euclidean distance of average visit spots and cosine similarity of other extracted features using K-means, KNN}
	\item {Developed a user info based recommendation system for future visiting prediction. Applied recsys package to train and achieved a top 20\% F1 score of 0.021 in POI recommendation.}
	\item {Predicted user gender profile according to heterogeneous data using models like SVM, Naive Bayes, Logistic Regression, and further analyzed and compared ensemble results of Random Forest, AdaBoost an d Voting and Bagging.}
	\item {Processed heterogeneous data from Weibo (Twitter of China) including basic user tweets contents and site access records into semi structured data by feature extraction.}
	\item {Applied packages of FudanNLP, jieba for Chinese text segmentation, word frequency statistics; PCA for dimension reduction; word2vec for embedding.}
      \end{cvitems}
    }
    
 \cventry
 {Fudan University} % Organization
    {Spread Trading Strategy Using Math Models} % Job title
{Advisor: Prof . Donghua Zhao} % Location
    {Mar. 2016 - May. 2016} % Date(s)
    {
      \begin{cvitems} % Description(s) of tasks/responsibilities
	\item {Implemented a framework of programmed trading system and simulated spread trading strategy.}
	\item {Forecasted future trend of short period stock prices of Copper-Zinc spread trading using models including Autoregressive Moving Average Model , Kalman Filtering and Hidden Markov Model.}
	\item {Applied forecasting model for new trading strategy and achieved annual profit rate of 21\%}
      \end{cvitems}
     }
    
\end{cventries}
