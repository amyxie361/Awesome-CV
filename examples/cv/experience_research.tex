%-------------------------------------------------------------------------------
%	SECTION TITLE
%-------------------------------------------------------------------------------
\cvsection{Research Experiences}


%-------------------------------------------------------------------------------
%	CONTENT
%-------------------------------------------------------------------------------
\begin{cventries}

%---------------------------------------------------------

 \cventry
    {University of Waterloo \& RSVP.ai}
    {Chatbot Module Development - Research Assistant} % Job title
    {Advisor:Prof. Ming Li, and Prof. Jimmy Lin} % Location
    {Sept. 2018 - Present} % Date(s)
    {
      \begin{cvitems} % Description(s) of tasks/responsibilities
        \item {Constructed an end-to-end question answering system that integrates BERT (\textbf{Tensorflow}) with the open-source Anserini (a \textbf{Lucene} IR toolkit) information retrieval toolkit both in English and Chinese and create new state of the art.}
        \item {Transferred code from \textbf{Tensorflow} implementation to \textbf{PyTorch}.}
        \item {Improved the system's performance by 10\% exact match rate on SQuAD 1.1 under open-domain setting using text augmentation and established new baselines on WebQuestion, CMRC and DRCD datasets.}
	\item {Tested the system's performance with \textbf{Elastic Search API} and provided real-time online service.}
        \item {Applied the system to domain specific document information retrieval.}
        \item {Applied a BERT based named entity recognition model to contract key information extraction. Implemented and compared bi-directional LSTM, VAE and GPT-2 as paraphrase generation models.}
      \end{cvitems}
    }
 
   \cventry
    {University of Waterloo}
    {Understanding How Human Generate Questions } % Job title
    {Advisor: Prof. Ori Friedman} % Location
    {Sept. 2019 - Presernt} % Date(s)
    {
      \begin{cvitems} % Description(s) of tasks/responsibilities
        \item {Surveyed the source of curiosity from psychology perspective.}
        \item {Surveyed the theory on linguistic perspective on question language.}
        \item {Analyzed the language features on two datasets: SQuAD and Quora Quesition Pair. }
      \end{cvitems}
    }
    
  \cventry
    {University of Waterloo}
    {{\href{https://github.com/amyxie361/CS651_Paper_recommendation}{\underline{Paper Recommendation using GraphX}}}} % Job title
    {Advisor: Prof. Jimmy Lin} % Location
    {Jan. 2019 - April 2019} % Date(s)
    {
      \begin{cvitems} % Description(s) of tasks/responsibilities
        \item {Applied \textbf{GraphX} to build an academic paper recommendation system.}
        \item {Implemented PageRank, keyword filtering, and pattern finding algorithms in GraphX and compared the framework against \textbf{MapReduce} on \textbf{Hadoop}.}
        \item {Applied the algorithm on a citation network to give recommendations of papers, taking users’ interest into account. }
      \end{cvitems}
    }

\cventry
    {University of Waterloo}
    {{\href{https://github.com/amyxie361/CS886}{\underline{Contextual Decomposition for Rationalizing LSTM Predictions}}}} % Job title
    {Advisor: {\href{https://cs.uwaterloo.ca/~y328yu/}{\underline{Prof. Yaoliang Yu}}}} % Location
    {Sept. 2018 - Nov 2018} % Date(s)
    {
      \begin{cvitems} % Description(s) of tasks/responsibilities
        \item {Decomposed and analyzed LSTM model in token level to understand the effectiveness source of the model based on entity detection task with \textbf{PyTorch}.}
        \item {Implemented and modified multi-view concept to understand the learned weight in the two-directions of LSTM model.}
        \item {Traced the source of the effective source of LSTM models and concluded the most effectiveness comes from embedding.}
      \end{cvitems}
    }
    
    \cventry
    {Machine Learning Algorithm Intership} % Job title
    {Yitu-Tech} % Organization
    {Shanghai, China} % Location
    {Feb. 2018 - June 2018} % Date(s)
    {
      \begin{cvitems} % Description(s) of tasks/responsibilities
       \item {Worked on the car detection project.}
        \item {Improved the  \textbf{Single Shot MultiBox Detector} model for object detection on car detection task.}
        \item {Implemented the HOG-SVM for digital recognition in car license detection.}
      \end{cvitems}
    }
    
      \cventry
     {Shanghai Key Laboratory of Intelligent Information Processing, Fudan University} 
    {BioASQ: Question Answering Based on Biomedical Paper Database} % Job title
    {Advisor: Prof . Shanfeng Zhu} % Location
    {July. 2017 - April. 2018} % Date(s)
    {
      \begin{cvitems} % Description(s) of tasks/responsibilities
	\item {Introduce occurrence possibility of words as the representation of answers for question answering.}
	\item {Adapted source code from TensorFlow for text processing including applying skip gram algorithm for word embedding matrix training.}
	\item {Constructed Bidirectional LSTM Recurrent Neural Networks under \textbf{Tensorflow} framework and introduced attention mechanism based on questions.}
	\item {Adjusted the model and achieved average Factoid MRR of 0.1615, List F measure of 0.1353.}
      \end{cvitems}
    }


    
\end{cventries}
